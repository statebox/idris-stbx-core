\documentclass{article}
%% ODER: format ==         = "\mathrel{==}"
%% ODER: format /=         = "\neq "
%
%
\makeatletter
\@ifundefined{lhs2tex.lhs2tex.sty.read}%
  {\@namedef{lhs2tex.lhs2tex.sty.read}{}%
   \newcommand\SkipToFmtEnd{}%
   \newcommand\EndFmtInput{}%
   \long\def\SkipToFmtEnd#1\EndFmtInput{}%
  }\SkipToFmtEnd

\newcommand\ReadOnlyOnce[1]{\@ifundefined{#1}{\@namedef{#1}{}}\SkipToFmtEnd}
\usepackage{amstext}
\usepackage{amssymb}
\usepackage{stmaryrd}
\DeclareFontFamily{OT1}{cmtex}{}
\DeclareFontShape{OT1}{cmtex}{m}{n}
  {<5><6><7><8>cmtex8
   <9>cmtex9
   <10><10.95><12><14.4><17.28><20.74><24.88>cmtex10}{}
\DeclareFontShape{OT1}{cmtex}{m}{it}
  {<-> ssub * cmtt/m/it}{}
\newcommand{\texfamily}{\fontfamily{cmtex}\selectfont}
\DeclareFontShape{OT1}{cmtt}{bx}{n}
  {<5><6><7><8>cmtt8
   <9>cmbtt9
   <10><10.95><12><14.4><17.28><20.74><24.88>cmbtt10}{}
\DeclareFontShape{OT1}{cmtex}{bx}{n}
  {<-> ssub * cmtt/bx/n}{}
\newcommand{\tex}[1]{\text{\texfamily#1}}	% NEU

\newcommand{\Sp}{\hskip.33334em\relax}


\newcommand{\Conid}[1]{\mathit{#1}}
\newcommand{\Varid}[1]{\mathit{#1}}
\newcommand{\anonymous}{\kern0.06em \vbox{\hrule\@width.5em}}
\newcommand{\plus}{\mathbin{+\!\!\!+}}
\newcommand{\bind}{\mathbin{>\!\!\!>\mkern-6.7mu=}}
\newcommand{\rbind}{\mathbin{=\mkern-6.7mu<\!\!\!<}}% suggested by Neil Mitchell
\newcommand{\sequ}{\mathbin{>\!\!\!>}}
\renewcommand{\leq}{\leqslant}
\renewcommand{\geq}{\geqslant}
\usepackage{polytable}

%mathindent has to be defined
\@ifundefined{mathindent}%
  {\newdimen\mathindent\mathindent\leftmargini}%
  {}%

\def\resethooks{%
  \global\let\SaveRestoreHook\empty
  \global\let\ColumnHook\empty}
\newcommand*{\savecolumns}[1][default]%
  {\g@addto@macro\SaveRestoreHook{\savecolumns[#1]}}
\newcommand*{\restorecolumns}[1][default]%
  {\g@addto@macro\SaveRestoreHook{\restorecolumns[#1]}}
\newcommand*{\aligncolumn}[2]%
  {\g@addto@macro\ColumnHook{\column{#1}{#2}}}

\resethooks

\newcommand{\onelinecommentchars}{\quad-{}- }
\newcommand{\commentbeginchars}{\enskip\{-}
\newcommand{\commentendchars}{-\}\enskip}

\newcommand{\visiblecomments}{%
  \let\onelinecomment=\onelinecommentchars
  \let\commentbegin=\commentbeginchars
  \let\commentend=\commentendchars}

\newcommand{\invisiblecomments}{%
  \let\onelinecomment=\empty
  \let\commentbegin=\empty
  \let\commentend=\empty}

\visiblecomments

\newlength{\blanklineskip}
\setlength{\blanklineskip}{0.66084ex}

\newcommand{\hsindent}[1]{\quad}% default is fixed indentation
\let\hspre\empty
\let\hspost\empty
\newcommand{\NB}{\textbf{NB}}
\newcommand{\Todo}[1]{$\langle$\textbf{To do:}~#1$\rangle$}

\EndFmtInput
\makeatother
%
%
%
%
%
%
% This package provides two environments suitable to take the place
% of hscode, called "plainhscode" and "arrayhscode". 
%
% The plain environment surrounds each code block by vertical space,
% and it uses \abovedisplayskip and \belowdisplayskip to get spacing
% similar to formulas. Note that if these dimensions are changed,
% the spacing around displayed math formulas changes as well.
% All code is indented using \leftskip.
%
% Changed 19.08.2004 to reflect changes in colorcode. Should work with
% CodeGroup.sty.
%
\ReadOnlyOnce{polycode.fmt}%
\makeatletter

\newcommand{\hsnewpar}[1]%
  {{\parskip=0pt\parindent=0pt\par\vskip #1\noindent}}

% can be used, for instance, to redefine the code size, by setting the
% command to \small or something alike
\newcommand{\hscodestyle}{}

% The command \sethscode can be used to switch the code formatting
% behaviour by mapping the hscode environment in the subst directive
% to a new LaTeX environment.

\newcommand{\sethscode}[1]%
  {\expandafter\let\expandafter\hscode\csname #1\endcsname
   \expandafter\let\expandafter\endhscode\csname end#1\endcsname}

% "compatibility" mode restores the non-polycode.fmt layout.

\newenvironment{compathscode}%
  {\par\noindent
   \advance\leftskip\mathindent
   \hscodestyle
   \let\\=\@normalcr
   \let\hspre\(\let\hspost\)%
   \pboxed}%
  {\endpboxed\)%
   \par\noindent
   \ignorespacesafterend}

\newcommand{\compaths}{\sethscode{compathscode}}

% "plain" mode is the proposed default.
% It should now work with \centering.
% This required some changes. The old version
% is still available for reference as oldplainhscode.

\newenvironment{plainhscode}%
  {\hsnewpar\abovedisplayskip
   \advance\leftskip\mathindent
   \hscodestyle
   \let\hspre\(\let\hspost\)%
   \pboxed}%
  {\endpboxed%
   \hsnewpar\belowdisplayskip
   \ignorespacesafterend}

\newenvironment{oldplainhscode}%
  {\hsnewpar\abovedisplayskip
   \advance\leftskip\mathindent
   \hscodestyle
   \let\\=\@normalcr
   \(\pboxed}%
  {\endpboxed\)%
   \hsnewpar\belowdisplayskip
   \ignorespacesafterend}

% Here, we make plainhscode the default environment.

\newcommand{\plainhs}{\sethscode{plainhscode}}
\newcommand{\oldplainhs}{\sethscode{oldplainhscode}}
\plainhs

% The arrayhscode is like plain, but makes use of polytable's
% parray environment which disallows page breaks in code blocks.

\newenvironment{arrayhscode}%
  {\hsnewpar\abovedisplayskip
   \advance\leftskip\mathindent
   \hscodestyle
   \let\\=\@normalcr
   \(\parray}%
  {\endparray\)%
   \hsnewpar\belowdisplayskip
   \ignorespacesafterend}

\newcommand{\arrayhs}{\sethscode{arrayhscode}}

% The mathhscode environment also makes use of polytable's parray 
% environment. It is supposed to be used only inside math mode 
% (I used it to typeset the type rules in my thesis).

\newenvironment{mathhscode}%
  {\parray}{\endparray}

\newcommand{\mathhs}{\sethscode{mathhscode}}

% texths is similar to mathhs, but works in text mode.

\newenvironment{texthscode}%
  {\(\parray}{\endparray\)}

\newcommand{\texths}{\sethscode{texthscode}}

% The framed environment places code in a framed box.

\def\codeframewidth{\arrayrulewidth}
\RequirePackage{calc}

\newenvironment{framedhscode}%
  {\parskip=\abovedisplayskip\par\noindent
   \hscodestyle
   \arrayrulewidth=\codeframewidth
   \tabular{@{}|p{\linewidth-2\arraycolsep-2\arrayrulewidth-2pt}|@{}}%
   \hline\framedhslinecorrect\\{-1.5ex}%
   \let\endoflinesave=\\
   \let\\=\@normalcr
   \(\pboxed}%
  {\endpboxed\)%
   \framedhslinecorrect\endoflinesave{.5ex}\hline
   \endtabular
   \parskip=\belowdisplayskip\par\noindent
   \ignorespacesafterend}

\newcommand{\framedhslinecorrect}[2]%
  {#1[#2]}

\newcommand{\framedhs}{\sethscode{framedhscode}}

% The inlinehscode environment is an experimental environment
% that can be used to typeset displayed code inline.

\newenvironment{inlinehscode}%
  {\(\def\column##1##2{}%
   \let\>\undefined\let\<\undefined\let\\\undefined
   \newcommand\>[1][]{}\newcommand\<[1][]{}\newcommand\\[1][]{}%
   \def\fromto##1##2##3{##3}%
   \def\nextline{}}{\) }%

\newcommand{\inlinehs}{\sethscode{inlinehscode}}

% The joincode environment is a separate environment that
% can be used to surround and thereby connect multiple code
% blocks.

\newenvironment{joincode}%
  {\let\orighscode=\hscode
   \let\origendhscode=\endhscode
   \def\endhscode{\def\hscode{\endgroup\def\@currenvir{hscode}\\}\begingroup}
   %\let\SaveRestoreHook=\empty
   %\let\ColumnHook=\empty
   %\let\resethooks=\empty
   \orighscode\def\hscode{\endgroup\def\@currenvir{hscode}}}%
  {\origendhscode
   \global\let\hscode=\orighscode
   \global\let\endhscode=\origendhscode}%

\makeatother
\EndFmtInput
%
\begin{document}

\section{Introduction}

\section{Category theory preliminaries}

Before starting to develop the Statebox typed core, we need to implement some basic categorical definitions which will allow 
us to stay as faithful as possible to the categorical model outlined in the Statebox Monograph. We start with the definition 
of category.

\begin{hscode}\SaveRestoreHook
\column{B}{@{}>{\hspre}l<{\hspost}@{}}%
\column{5}{@{}>{\hspre}l<{\hspost}@{}}%
\column{E}{@{}>{\hspre}l<{\hspost}@{}}%
\>[B]{}\mathbf{module}\;\Conid{Category}{}\<[E]%
\\[\blanklineskip]%
\>[B]{}\mathbin{\%}\Varid{access}\;\Varid{public}\;\Varid{export}{}\<[E]%
\\
\>[B]{}\mathbin{\%}\mathbf{default}\;\Varid{total}{}\<[E]%
\\[\blanklineskip]%
\>[B]{}\mathbf{data}\;\Conid{Category}\mathbin{:}(\Varid{obj}\mathbin{:}\Conid{Type})\to (\Varid{mor}\mathbin{:}\Varid{obj}\to \Varid{obj}\to \Conid{Type})\to \Conid{Type}\;\mathbf{where}{}\<[E]%
\\
\>[B]{}\Conid{MkCategory}\mathbin{:}{}\<[E]%
\\
\>[B]{}\hsindent{5}{}\<[5]%
\>[5]{}\{\mskip1.5mu \Varid{obj}\mathbin{:}\Conid{Type}\mskip1.5mu\}{}\<[E]%
\\
\>[B]{}\hsindent{5}{}\<[5]%
\>[5]{}\to \{\mskip1.5mu \Varid{mor}\mathbin{:}\Varid{obj}\to \Varid{obj}\to \Conid{Type}\mskip1.5mu\}{}\<[E]%
\\
\>[B]{}\hsindent{5}{}\<[5]%
\>[5]{}\to (\Varid{identity}\mathbin{:}(\Varid{a}\mathbin{:}\Varid{obj})\to \Varid{mor}\;\Varid{a}\;\Varid{a}){}\<[E]%
\\
\>[B]{}\hsindent{5}{}\<[5]%
\>[5]{}\to (\Varid{compose}\mathbin{:}(\Varid{a},\Varid{b},\Varid{c}\mathbin{:}\Varid{obj})\to (\Varid{f}\mathbin{:}\Varid{mor}\;\Varid{a}\;\Varid{b})\to (\Varid{g}\mathbin{:}\Varid{mor}\;\Varid{b}\;\Varid{c})\to \Varid{mor}\;\Varid{a}\;\Varid{c}){}\<[E]%
\\
\>[B]{}\hsindent{5}{}\<[5]%
\>[5]{}\to \Conid{Category}\;\Varid{obj}\;\Varid{mor}{}\<[E]%
\ColumnHook
\end{hscode}\resethooks

We define a category as a type. It requires to specify a set of objects and, for each object, a set of morphisms 
implemented as functions that take two objects in input and return a type as output. Category has one constructor,
which specifies identities on objects as functions (to each object corresponds an identiy morphism).
Similarly it specifies compositions of morphisms (provided three objects and two morphism between them with matching
domain/codomain, a composition object is produced).

At the moment nothing ensures that Category has is a category, because identity and associativity laws are not enforced.
To solve this, we start by implementing the laws:

\begin{hscode}\SaveRestoreHook
\column{B}{@{}>{\hspre}l<{\hspost}@{}}%
\column{3}{@{}>{\hspre}l<{\hspost}@{}}%
\column{6}{@{}>{\hspre}l<{\hspost}@{}}%
\column{E}{@{}>{\hspre}l<{\hspost}@{}}%
\>[B]{}\Conid{LeftIdentity}\mathbin{:}{}\<[E]%
\\
\>[B]{}\hsindent{6}{}\<[6]%
\>[6]{}\{\mskip1.5mu \Varid{obj}\mathbin{:}\Conid{Type}\mskip1.5mu\}{}\<[E]%
\\
\>[B]{}\hsindent{3}{}\<[3]%
\>[3]{}\to \{\mskip1.5mu \Varid{mor}\mathbin{:}\Varid{obj}\to \Varid{obj}\to \Conid{Type}\mskip1.5mu\}{}\<[E]%
\\
\>[B]{}\hsindent{3}{}\<[3]%
\>[3]{}\to \{\mskip1.5mu \Varid{a},\Varid{b}\mathbin{:}\Varid{obj}\mskip1.5mu\}{}\<[E]%
\\
\>[B]{}\hsindent{3}{}\<[3]%
\>[3]{}\to (\Varid{f}\mathbin{:}\Varid{mor}\;\Varid{a}\;\Varid{b}){}\<[E]%
\\
\>[B]{}\hsindent{3}{}\<[3]%
\>[3]{}\to \Conid{Category}\;\Varid{obj}\;\Varid{mor}{}\<[E]%
\\
\>[B]{}\hsindent{3}{}\<[3]%
\>[3]{}\to \Conid{Type}{}\<[E]%
\\
\>[B]{}\Conid{LeftIdentity}\;\{\mskip1.5mu \Varid{obj}\mskip1.5mu\}\;\{\mskip1.5mu \Varid{mor}\mskip1.5mu\}\;\{\mskip1.5mu \Varid{a}\mskip1.5mu\}\;\{\mskip1.5mu \Varid{b}\mskip1.5mu\}\;\Varid{f}\;(\Conid{MkCategory}\;\Varid{identity}\;\Varid{compose})\mathrel{=}{}\<[E]%
\\
\>[B]{}\hsindent{3}{}\<[3]%
\>[3]{}\Varid{compose}\;\Varid{a}\;\Varid{a}\;\Varid{b}\;(\Varid{identity}\;\Varid{a})\;\Varid{f}\mathrel{=}\Varid{f}{}\<[E]%
\ColumnHook
\end{hscode}\resethooks

\begin{hscode}\SaveRestoreHook
\column{B}{@{}>{\hspre}l<{\hspost}@{}}%
\column{3}{@{}>{\hspre}l<{\hspost}@{}}%
\column{6}{@{}>{\hspre}l<{\hspost}@{}}%
\column{E}{@{}>{\hspre}l<{\hspost}@{}}%
\>[B]{}\Conid{RightIdentity}\mathbin{:}{}\<[E]%
\\
\>[B]{}\hsindent{6}{}\<[6]%
\>[6]{}\{\mskip1.5mu \Varid{obj}\mathbin{:}\Conid{Type}\mskip1.5mu\}{}\<[E]%
\\
\>[B]{}\hsindent{3}{}\<[3]%
\>[3]{}\to \{\mskip1.5mu \Varid{mor}\mathbin{:}\Varid{obj}\to \Varid{obj}\to \Conid{Type}\mskip1.5mu\}{}\<[E]%
\\
\>[B]{}\hsindent{3}{}\<[3]%
\>[3]{}\to \{\mskip1.5mu \Varid{a},\Varid{b}\mathbin{:}\Varid{obj}\mskip1.5mu\}{}\<[E]%
\\
\>[B]{}\hsindent{3}{}\<[3]%
\>[3]{}\to (\Varid{f}\mathbin{:}\Varid{mor}\;\Varid{a}\;\Varid{b}){}\<[E]%
\\
\>[B]{}\hsindent{3}{}\<[3]%
\>[3]{}\to \Conid{Category}\;\Varid{obj}\;\Varid{mor}{}\<[E]%
\\
\>[B]{}\hsindent{3}{}\<[3]%
\>[3]{}\to \Conid{Type}{}\<[E]%
\\
\>[B]{}\Conid{RightIdentity}\;\{\mskip1.5mu \Varid{obj}\mskip1.5mu\}\;\{\mskip1.5mu \Varid{mor}\mskip1.5mu\}\;\{\mskip1.5mu \Varid{a}\mskip1.5mu\}\;\{\mskip1.5mu \Varid{b}\mskip1.5mu\}\;\Varid{f}\;(\Conid{MkCategory}\;\Varid{identity}\;\Varid{compose})\mathrel{=}{}\<[E]%
\\
\>[B]{}\hsindent{3}{}\<[3]%
\>[3]{}\Varid{compose}\;\Varid{a}\;\Varid{b}\;\Varid{b}\;\Varid{f}\;(\Varid{identity}\;\Varid{b})\mathrel{=}\Varid{f}{}\<[E]%
\ColumnHook
\end{hscode}\resethooks

The left identity law takes a Category and one of its morphisms, and produces an equation proving that composing the morphism 
on the left with the identity morphism on its domain amounts to doing nothing. Right identity law is defined analogously:


It remains to implement associativity, which is done as follows:

\begin{hscode}\SaveRestoreHook
\column{B}{@{}>{\hspre}l<{\hspost}@{}}%
\column{3}{@{}>{\hspre}l<{\hspost}@{}}%
\column{6}{@{}>{\hspre}l<{\hspost}@{}}%
\column{E}{@{}>{\hspre}l<{\hspost}@{}}%
\>[B]{}\Conid{Associativity}\mathbin{:}{}\<[E]%
\\
\>[B]{}\hsindent{6}{}\<[6]%
\>[6]{}\{\mskip1.5mu \Varid{obj}\mathbin{:}\Conid{Type}\mskip1.5mu\}{}\<[E]%
\\
\>[B]{}\hsindent{3}{}\<[3]%
\>[3]{}\to \{\mskip1.5mu \Varid{mor}\mathbin{:}\Varid{obj}\to \Varid{obj}\to \Conid{Type}\mskip1.5mu\}{}\<[E]%
\\
\>[B]{}\hsindent{3}{}\<[3]%
\>[3]{}\to \{\mskip1.5mu \Varid{a},\Varid{b},\Varid{c},\Varid{d}\mathbin{:}\Varid{obj}\mskip1.5mu\}{}\<[E]%
\\
\>[B]{}\hsindent{3}{}\<[3]%
\>[3]{}\to \{\mskip1.5mu \Varid{f}\mathbin{:}\Varid{mor}\;\Varid{a}\;\Varid{b}\mskip1.5mu\}{}\<[E]%
\\
\>[B]{}\hsindent{3}{}\<[3]%
\>[3]{}\to \{\mskip1.5mu \Varid{g}\mathbin{:}\Varid{mor}\;\Varid{b}\;\Varid{c}\mskip1.5mu\}{}\<[E]%
\\
\>[B]{}\hsindent{3}{}\<[3]%
\>[3]{}\to \{\mskip1.5mu \Varid{h}\mathbin{:}\Varid{mor}\;\Varid{c}\;\Varid{d}\mskip1.5mu\}{}\<[E]%
\\
\>[B]{}\hsindent{3}{}\<[3]%
\>[3]{}\to \Conid{Category}\;\Varid{obj}\;\Varid{mor}{}\<[E]%
\\
\>[B]{}\hsindent{3}{}\<[3]%
\>[3]{}\to \Conid{Type}{}\<[E]%
\\
\>[B]{}\Conid{Associativity}\;\{\mskip1.5mu \Varid{obj}\mskip1.5mu\}\;\{\mskip1.5mu \Varid{mor}\mskip1.5mu\}\;\{\mskip1.5mu \Varid{a}\mskip1.5mu\}\;\{\mskip1.5mu \Varid{b}\mskip1.5mu\}\;\{\mskip1.5mu \Varid{c}\mskip1.5mu\}\;\{\mskip1.5mu \Varid{d}\mskip1.5mu\}\;\{\mskip1.5mu \Varid{f}\mskip1.5mu\}\;\{\mskip1.5mu \Varid{g}\mskip1.5mu\}\;\{\mskip1.5mu \Varid{h}\mskip1.5mu\}\;(\Conid{MkCategory}\;\Varid{identity}\;\Varid{compose})\mathrel{=}{}\<[E]%
\\
\>[B]{}\hsindent{3}{}\<[3]%
\>[3]{}\Varid{compose}\;\Varid{a}\;\Varid{b}\;\Varid{d}\;\Varid{f}\;(\Varid{compose}\;\Varid{b}\;\Varid{c}\;\Varid{d}\;\Varid{g}\;\Varid{h})\mathrel{=}\Varid{compose}\;\Varid{a}\;\Varid{c}\;\Varid{d}\;(\Varid{compose}\;\Varid{a}\;\Varid{b}\;\Varid{c}\;\Varid{f}\;\Varid{g})\;\Varid{h}{}\<[E]%
\ColumnHook
\end{hscode}\resethooks

Unsurprisingly, associativity takes a category and produces a proof that for each triple of morphisms with matching domains and
codomains, the order of composition does not matter.

We can now combine the Category type with the laws just implemented to obtain a VerifiedCategory type, which is a Category
obeying the unit and associativity laws:

\begin{hscode}\SaveRestoreHook
\column{B}{@{}>{\hspre}l<{\hspost}@{}}%
\column{3}{@{}>{\hspre}l<{\hspost}@{}}%
\column{5}{@{}>{\hspre}l<{\hspost}@{}}%
\column{8}{@{}>{\hspre}l<{\hspost}@{}}%
\column{E}{@{}>{\hspre}l<{\hspost}@{}}%
\>[B]{}\mathbf{data}\;\Conid{VerifiedCategory}\mathbin{:}(\Varid{obj}\mathbin{:}\Conid{Type})\to (\Varid{mor}\mathbin{:}\Varid{obj}\to \Varid{obj}\to \Conid{Type})\to \Conid{Type}\;\mathbf{where}{}\<[E]%
\\
\>[B]{}\hsindent{3}{}\<[3]%
\>[3]{}\Conid{MkVerifiedCategory}\mathbin{:}{}\<[E]%
\\
\>[3]{}\hsindent{5}{}\<[8]%
\>[8]{}(\Varid{cat}\mathbin{:}\Conid{Category}\;\Varid{obj}\;\Varid{mor}){}\<[E]%
\\
\>[3]{}\hsindent{2}{}\<[5]%
\>[5]{}\to ((\Varid{a},\Varid{b}\mathbin{:}\Varid{obj})\to (\Varid{f}\mathbin{:}\Varid{mor}\;\Varid{a}\;\Varid{b})\to \Conid{LeftIdentity}\;\Varid{f}\;\Varid{cat}){}\<[E]%
\\
\>[3]{}\hsindent{2}{}\<[5]%
\>[5]{}\to ((\Varid{a},\Varid{b}\mathbin{:}\Varid{obj})\to (\Varid{f}\mathbin{:}\Varid{mor}\;\Varid{a}\;\Varid{b})\to \Conid{RightIdentity}\;\Varid{f}\;\Varid{cat}){}\<[E]%
\\
\>[3]{}\hsindent{2}{}\<[5]%
\>[5]{}\to ((\Varid{a},\Varid{b},\Varid{c},\Varid{d}\mathbin{:}\Varid{obj})\to (\Varid{f}\mathbin{:}\Varid{mor}\;\Varid{a}\;\Varid{b})\to (\Varid{g}\mathbin{:}\Varid{mor}\;\Varid{b}\;\Varid{c})\to (\Varid{h}\mathbin{:}\Varid{mor}\;\Varid{c}\;\Varid{d})\to \Conid{Associativity}\;\{\mskip1.5mu \Varid{f}\mskip1.5mu\}\;\{\mskip1.5mu \Varid{g}\mskip1.5mu\}\;\{\mskip1.5mu \Varid{h}\mskip1.5mu\}\;\Varid{cat}){}\<[E]%
\\
\>[3]{}\hsindent{2}{}\<[5]%
\>[5]{}\to \Conid{VerifiedCategory}\;\Varid{obj}\;\Varid{mor}{}\<[E]%
\ColumnHook
\end{hscode}\resethooks

As you can see, here the constructor requires that morphism obey the associativity and unit laws we defined before. We conclude the
section by defining InnerCategory, which takes a VerifiedCategory and strips the verified part away, producing the underlying
Category structure. InnerCategory greatly simplifies life when we have to define more complicated mathematical objects such as functors.

\begin{hscode}\SaveRestoreHook
\column{B}{@{}>{\hspre}l<{\hspost}@{}}%
\column{E}{@{}>{\hspre}l<{\hspost}@{}}%
\>[B]{}\Varid{innerCategory}\mathbin{:}\Conid{VerifiedCategory}\;\Varid{obj}\;\Varid{mor}\to \Conid{Category}\;\Varid{obj}\;\Varid{mor}{}\<[E]%
\\
\>[B]{}\Varid{innerCategory}\;(\Conid{MkVerifiedCategory}\;\Varid{cat}\;\anonymous \;\anonymous \;\anonymous )\mathrel{=}\Varid{cat}{}\<[E]%
\ColumnHook
\end{hscode}\resethooks

\end{document}
